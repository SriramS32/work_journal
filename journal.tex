\documentclass[a4paper]{article}

%% Language and font encodings
\usepackage[english]{babel}
\usepackage[utf8x]{inputenc}
\usepackage[T1]{fontenc}

%% Sets page size and margins
\usepackage[a4paper,top=3cm,bottom=2cm,left=3cm,right=3cm,marginparwidth=1.75cm]{geometry}

%% Useful packages
\usepackage{amsmath}
\usepackage{graphicx}
\usepackage[colorinlistoftodos]{todonotes}
\usepackage[colorlinks=true, allcolors=blue]{hyperref}

\title{Work/ Idea Journal}
\author{Sriram Somasundaram}
\date{}

\begin{document}
\maketitle

\section{Introduction}

The following is just a journal of my daily activities and thoughts. Many ideas will be cursory without background knowledge, which I may research thoroughly and pursue more in depth.

\section{Days}

\subsection{July 26, 2017}
\subparagraph{Work}
Spent about twenty minutes today with some friends figuring out the best way to implement a n-dimensional array in C using dynamic allocation and trying to make the memory contiguous and easily accessible. I was also trying to understand how Mel-Frequency Cepstral Coefficients (MFCCs) are generated for audio parametrization in speech processing. A lot of the signal processing to parametrize the audio is motivated by how we hear in the cochlea, but the last ste is a Discrete Cosine Transform. Apparently the DCT is similar to Principal Component Analysis, which both involve something called a Karhunen-Loève transform. Somehow the eigenvectors of the covariance matrix are involved, and I was trying to make sense of what those meant and how using those as a basis minimized mean squared error. Also, I tried out some of the TFLearn library (higher-level API for Tensorflow) just for classification using DNNs and tried to do sequence generation.
\subparagraph{Ideas}
Maybe make a web app template with database support, auth, which uses a couple of modern tools in a pipeline for startups and new developers to use. LSTM for web syntax then link of text descriptions of websites and try to be able to generate websites (HTML, CSS, JS) given a text description of the desired site. Look at what is being done for generating images from text (\href{https://arxiv.org/pdf/1605.05396.pdf}{Ex 1} and \href{https://arxiv.org/pdf/1511.02793.pdf}{2})


\subsection{July 27, 2017}
\subparagraph{Work}
Checked out the tflearn implementation for LSTM and text generation. Thought about how to use the LSTM as a language model that learns word and character sequence probabilities. Needed to refer to Parallel Phone Recognition and Language modeling. Also, was trying to learn about latent semantic analysis and latent dirichlet allocation.
\subparagraph{Ideas}
Gather the titles and topics of arXiv papers and try to predict the direction research is heading to. May have to build some semantic model along with a sequence one.


\end{document}